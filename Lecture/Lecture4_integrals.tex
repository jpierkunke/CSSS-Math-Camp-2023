\documentclass{beamer}
%\documentclass{beamer}
\usetheme{umbc2}  % name of theme substututed here
\usepackage{hyperref}
\usepackage{pause}

\usepackage{}

\makeatletter
\setbeamertemplate{footline}
{
  \leavevmode%
  \hbox{%
  
    \begin{beamercolorbox}[wd=.333333\paperwidth,ht=2.25ex,dp=1ex,center]{title in head/foot}%
    \usebeamerfont{title in head/foot}\insertshorttitle
  \end{beamercolorbox}%
  \begin{beamercolorbox}[wd=.333333\paperwidth,ht=2.25ex,dp=1ex,center]{author in head/foot}%
    \usebeamerfont{author in head/foot}\insertshortauthor~~\beamer@ifempty{\insertshortinstitute}{}{(\insertshortinstitute)}
  \end{beamercolorbox}%
  \begin{beamercolorbox}[wd=.333333\paperwidth,ht=2.25ex,dp=1ex,right]{date in head/foot}%
    \usebeamerfont{date in head/foot}\insertshortdate{}\hspace*{2em}
    \insertframenumber\hspace*{2ex} 
  \end{beamercolorbox}}%
  \vskip0pt%
}
\makeatother

% You can set the default options using \SweaveOpts


% default options for includegraphics used when "fig=true"
\setkeys{Gin}{keepaspectratio=true, height=0.85\textheight}

% Get rid of the extra navigation symbols that beamer puts in your slides
\beamertemplatenavigationsymbolsempty


\begin{document}
\title[CSSS Math Camp - Lecture 4]{CSSS Math Camp Lecture 4}
\subtitle{Integral Calculus \\ \small{Authored by: Laina Mercer, PhD}}
\author{Erin Lipman and Jess Kunke}
\institute[UW]{Department of Statistics}
\date{September 14, 2023}

\begin{frame}[plain] 
  \titlepage
\end{frame}



\begin{frame}{Lecture 4: Integral Calculus}

\begin{itemize}
\item Motivation for Integrals
\item Definition of integration
\item Rules of Integration
\end{itemize}

\end{frame}



\begin{frame}{Differentiation Example}{distance, velocity, acceleration}
Let's take $d$=distance, $v$=velocity, $a$=acceleration.  You may remember from physics, the distance traveled after time $t$
\[
d(t)=\frac{a}{2}t^2
\]
%\pause
The velocity at any time $t$ is the instantaneous rate of change of the distance, $v(t)=d'(t)$:

\[
v(t)=2\cdot \frac{a}{2}t=at
\]
%\pause
The acceleration at any time $t$ is the instantaneous rate of change of the velocity, $a(t)=v'(t)=d''(t)$:
\[
a(t)=a
\]
\end{frame}

\begin{frame}{Distance}
\begin{figure}[h!]\centering
    \includegraphics[width=1.0\textwidth]{distance.pdf}
    \caption{Distance over time, when $a(t)=2$, $v(t)=2t$, and $d(t)=t^2$.}
\end{figure}
\end{frame}

\begin{frame}{Velocity}
\begin{figure}[h!]\centering
    \includegraphics[width=1.0\textwidth]{velocity.pdf}
    \caption{Velocity over time, when $a(t)=2$, $v(t)=2t$, and $d(t)=t^2$.}
\end{figure}
\end{frame}

\begin{frame}{Acceleration}
%$$a(t)=2$$
\begin{figure}[h!]\centering
    \includegraphics[width=1.0\textwidth]{acceleration.pdf}
    \caption{Acceleration over time, when $a(t)=2$, $v(t)=2t$, and $d(t)=t^2$.}
\end{figure}
\end{frame}





\begin{frame}{What is the velocity at t=3 when a=2?}
%\pause
We know that $v(t)=2t$, so clearly $$v(3)=2\cdot 3=6.$$

%\pause 
However we can also find the velocity, by looking at the area under the acceleration curve from $t=0$ to $t=3$.  This would just be the area of a rectangle (\texttt{base X height}), $$(3-0)\cdot 2=3\cdot2=6.$$

\end{frame}

\begin{frame}{What is the velocity at t=3 when a=2?}

\begin{figure}[h!]\centering
    \includegraphics[width=1.0\textwidth]{acc_veloc.pdf}
   % \caption{Distance over time, when $a=2$.}
\end{figure}
\end{frame}


\begin{frame}{What is the distance at t=3 when a=2?}

We know that $ d(t)=2/2t^2=t^2$, so clearly  $$d(3)=3^2=9.$$  

%\pause 
However we can also find the distance, by looking at the area under the velocity curve from $t=0$ to $t=3$.  This would just be the area of a triangle ($1/2 $\texttt{ X base X height}), $$1/2\cdot (3-0)\cdot 6=3/2\cdot 6 =18/2=9.$$

\end{frame}


\begin{frame}{What is the distance at t=3 when a=2?}
\begin{figure}[h!]\centering
    \includegraphics[width=1.0\textwidth]{veloc_dist.pdf}
   % \caption{Distance over time, when $a=2$.}
\end{figure}
\end{frame}


\begin{frame}{Motivation for Integrals in Statistics}
\begin{figure}[h!]\centering
    \includegraphics[width=0.95\textwidth]{normal.pdf}
    \caption{Standard Normal Density (N(0,1)).  Approximately 68\% of the probability lies within 1 standard deviation and 95\% within 2 standard deviations. The area under the whole curve (from $-\infty$ to $\infty$) is 1.}
\end{figure}
\end{frame}

\begin{frame}{Motivation for Integrals in Statistics}
Integral caclulus...
\begin{itemize}
\item is a tool for computing areas under curves.
%\item can be used to compute percentile rankings.
\item is used to compute probabilities of events.
\item will be needed to compute the expected values and variance of probability distributions.
\item is heavily used in statistical theory!
\end{itemize}
\end{frame}


\begin{frame}{Motivation for Integrals in Statistics}{What if we wanted to find the area under the curve from -2 to -0.5?}
\begin{figure}[h!]\centering
    \includegraphics[width=0.95\textwidth]{normal_example.pdf}
  %  \caption{Standard Normal Density (N(0,1)).  Approximately 68\% of the probability lies within 1 standard deviation and 95\% within 2 standard deviations.}
\end{figure}
\end{frame}


\begin{frame}{Motivation for Integrals in Statistics}{We could approximate with rectangles or trapezoids.  Narrower rectangles would give better approximations.}
\begin{figure}[h!]\centering
    \includegraphics[width=0.95\textwidth]{normal_boxes.pdf}
  %  \caption{Standard Normal Density (N(0,1)).  Approximately 68\% of the probability lies within 1 standard deviation and 95\% within 2 standard deviations.}
\end{figure}
\end{frame}

\begin{frame}{Integration}
The area under a curve is written:
\[
\int\limits_a^bf(x)dx
\]

This formula is called the \textit{definite integral} of $f(x)$ from $a$ to $b$.  

\vspace{3mm}
%Here $a$ and $b$ are our endpoints of interest.  There is a direct relationship (via the Fundamental Theorem of Calculus) with finding a derivative.  You can think of the integral as the `opposite' of the derivative.
Here $a$ and $b$ are our endpoints of interest.   You can think of the integral as the `opposite' of the derivative.

\end{frame}

\begin{frame}{Integration}

More specifically,
\[
\int\limits_a^bf(x)dx=F(b)-F(a) \textrm{ where } F'(x)=f(x)
\]

$F(x)$ is called the \textit{indefinite integral} of $f(x)$. The important relationships between derivatives and integrals are:

\[
F'(x)=f(x) \;\;\textrm{        \&      } \int f(x)dx=F(x)
\]

\end{frame}

\begin{frame}{What is an integral?}
You can think of integrating as looking at a derivative and trying to find the original function.
\vspace{3mm}

%\pause
\begin{itemize}
\item $\int 3 dx$.
What function has a derivate equal to 3? %\pause $3x$. \pause
\item $\int 2x dx$. %\pause 
What function has a derivate equal to $2x$? %\pause $x^2$. \pause
\item $\int e^x dx$. %\pause 
What function has a derivate equal to $e^x$? %\pause $e^x$.
\end{itemize}
\vspace{2mm}
In practice, you don't have to search for the right function. We have handy shortcuts (rules).

\end{frame}

\begin{frame}{Integration Rules}{Integrating a Constant}
\[
\int c dx=cx
\]
%\pause
Examples:
\begin{itemize}
\item $\int 1dx=$%\pause x$ \pause
\vspace{2mm}
\item $\int 6dx=$%\pause 6x$ \pause
\vspace{2mm}
\item $\int ydx=$%\pause yx$
\end{itemize}
\end{frame}


\begin{frame}{Integration Rules}{Integrating a Power of $x$}
\[
\int x^n dx=\frac{1}{n+1}x^{n+1}
\] %\pause
Examples:
\begin{itemize}
\item $\int xdx=$%\pause \frac{1}{2}x^2$ \pause
%\item $\int 4x^2dx=\pause 4\left( \frac{1}{3}x^3 \right)  \pause = \frac{4}{3}x^3  $ \pause
\vspace{2mm}
\item $\int \frac{1}{x^2}dx=$%\pause \int x^{-2} dx=\pause \frac{1}{-1}x^{-1}\pause= -\frac{1}{x}$ \pause
%\item $\int c ydy=$%\pause \mu \left( \frac{1}{2}y^2 \right)  \pause = \frac{\mu}{2}y^2  $ 
\end{itemize}
\end{frame}



\begin{frame}{Integration Rules}{Integrating an Exponential and Logarithmic Functions}
Exponential:
\[
\int e^x dx=e^x
\]
\vspace{3mm}

(Natural) Logarithm:
\[
\int \frac{1}{x} dx=log(x)
\]


\end{frame}



%\begin{frame}{Integration Rules}{Basic Trigonometric Functions}
%Remember, $\frac{d}{dx}cos(x)=-sin(x)$, thus
%\[
%\int sin(x) dx=-cos(x)
%\]
%\vspace{3mm}

%and $\frac{d}{dx}sin(x)=cos(x)$, thus
%\[
%\int cos(x) dx=sin(x).
%\]
%\end{frame}

\begin{frame}{Integration Rules}{Multiple of a Function}
\[
\int af(x)dx=a\cdot \int f(x)dx=aF(x)
\] %\pause
Examples:
\begin{itemize}
\item $\int 4x^2dx= $%4\int x^2dx=4\left( \frac{1}{3}x^3 \right)  = \frac{4}{3}x^3  $ 
\vspace{2mm}
\item $\int \frac{3}{x^2}dx=$% 3\int\frac{1}{x^2}dx=3\int x^{-2} dx= \frac{3}{-1}x^{-1}= -\frac{3}{x}$ %pause
\vspace{2mm}
\item $\int \mu ydy= $%\mu \int ydy=\mu \left( \frac{1}{2}y^2 \right) = \frac{\mu}{2}y^2  $ 
\end{itemize}
\end{frame}


\begin{frame}{Integration Rules}{Sums of Functions}
\[
\int \left(f(x)+g(x) \right)dx=\int f(x)dx+\int g(x)dx=F(x)+G(x)
\] %\pause
Examples:
\begin{itemize}
\item $\int \left(4x+3x^2\right)dx$%=\int 4x dx+ \int 3x^2dx = 4\int x dx + 3\int
 %x^2dx= 4\cdot\frac{1}{2}x^2+3\cdot \frac{1}{3} x^3=\pause 2x^2+x^3  $ 
\vspace{10mm}
\item $\int \left(e^x-\frac{2}{x}\right)dx=$%\int e^xdx-2\int \frac{1}{x}dx= %e^x-2log(x)$ 
\end{itemize}
\end{frame}


\begin{frame}{Integration Rules}{$u$-substitution}
Sometimes the function we are integrating is similar to a simpler function with an easy derivative.\\
\vspace{3mm}
%\pause

For example, $\int \frac{1}{1-x}dx$ is similar to $\int \frac{1}{x}dx$ which we know is $log(x)$.  Similar to the chain rule, we can think about functions within functions.

\vspace{3mm}
\pause
Let's set $u=1-x$.  If we differentiate the left with respect to $u$ and the right with respect to $x$ we have $du=-1dx$.  Solving for $dx$ we have $dx=-1du$.  Now we can substitute these values into our original integral.

\end{frame}

\begin{frame}{Integration Rules}{$u$-substitution continued}
Example: Find $\int \frac{1}{1-x}dx$ using the substitution $u=1-x$ (and so $du = -dx$).
\vspace{2mm}

First substitute $u=1-x$ and $du=-dx$ into the integral:
\[
\int \frac{1}{1-x}dx=\int\frac{1}{u}\cdot (-1)du=-1\int \frac{1}{u}du
\]
\pause
Now let's take the integral with respect to $u$:
\[
-1\int \frac{1}{u}du=-log(u)
\]
\pause
Then we can plug in the value for $u=1-x$:
\[
-log(u)=-log(1-x)
\]
\end{frame}

\begin{frame}{Integration Rules}{$u$-substitution continued}
Example: 
\[
\int (2x+4)^3dx\] %\pause
%We can take $u=2x+4$.  Then $du=2dx$ or $\frac{1}{2}du=dx$.  
\vspace{50mm}

%When we make the substitutions in our integral we have:
%\[
%\int (2x+4)^3dx=\int u^3\cdot \frac{1}{2}du=\frac{1}{2}\int u^3du
%\] 
%Now we have an integral we can easily compute 
%\[\frac{1}{2}\int u^3du=\frac{1}{2}\cdot \frac{1}{4}u^4=\frac{1}{8}u^4\]
% and then we just need to substitute back in for the functions of $x$.
% \[
%\int (2x+4)^3dx=\frac{1}{2}\int u^3du=\frac{1}{8}u^4=\frac{1}{8}(2x+4)^4
%\]
\end{frame}


\begin{frame}{Finding Definite Integrals}
Often we will be interested in knowing the exact area under the curve $f(x)$, not just the function $F(x)$.
\[
\int\limits_a^bf(x)dx=F(x)|_a^b=F(b)-F(a)
\]



Examples:
\begin{itemize}
\item$ \int\limits_{0}^1x^2dx=$%\pause \frac{1}{3}x^3|_0^1=\pause \frac{1}{3}1^3-\frac{1}{3}0^3=\frac{1}{3}$\pause
\vspace{2mm}
\item$ \int\limits_{0}^{\infty}e^{-x}dx=$%\pause -e^{-x}|_0^{\infty}=\pause -e^{-\infty}--e^0=-\frac{1}{e^{\infty}}+e^0=1$
\vspace{2mm}
\item$ \int\limits_{2}^{8}\frac{1}{x}dx=$%\pause log(x)|_2^{8}=\pause log(8)-log(2)=log\left( \frac{8}{2}\right)=log(4)$
\end{itemize}

\end{frame}

\begin{frame}{Integration Example}{distance, velocity, acceleration}

Back to our original example, with $a=2$.  The velocity at time $t=3$ is the definite integral of of the acceleration, $v(3)=\int\limits_0^{3} a(t)dt$:

\[
v(3)=\int\limits_0^{3}  2 dt=2t|_0^3=2\cdot 3-2\cdot 0=(3-0)\cdot 2=6
\]
\pause
Similarly, the distance at time $t=3$ is the definite integral of of the velocity, $d(3)=\int\limits_0^{3} v(t)dt$:
\[
d(3)=\int\limits_0^{3} v(t)dt=\int\limits_0^{3} 2tdt=t^2|_0^{3}=3^2-0^2=9
\]

\end{frame}


\begin{frame}{Example}
\[
\int\limits_0^3 e^{x/3}dx %\pause
\]
%We could take $u=x/3$.  Then $du=1/3dx$ and $3du=dx$. 
\vspace{60mm}

%When we substitute in for $u$ and $dx$ it is important to note that we must also %substitute in for our limits of integration.  The lower value $u=0/3=0$ and the upper %value would be $u=3/3=1$.
%\[
%\int\limits_0^3 e^{x/3}dx=\int\limits_0^1 e^{u}\cdot 3du=3\int\limits_0^1 e^{u}du=3e^u|_0^1=3(e^1-e^0)=3(e-1)
%\]
\end{frame}

%
%\begin{frame}{Integration by Parts}
%
%For complicated functions it is often handy to decompose the function into two parts.
% \[
% \int f(x)dx\pause= \int g(x)\cdot h'(x)dx \pause =g(x)h(x)-\int h(x)\cdot g'(x)dx\pause
% \]
% Example:
% \[
% \int xe^xdx
% \]
% \pause
%$$g(x) = x, \hspace{.3in} h^{'}(x) = e^x dx$$\pause
%$$g^{'}(x) = 1, \hspace{.3in} h(x) = e^x $$ \pause
%\[
%\int xe^xdx=x\cdot e^x - \int e^x \cdot 1 dx\pause=xe^x-e^x=e^x(x-1)
%\]
%\end{frame}
%
%\begin{frame}{Integration by Parts}{Beta Distribution Example}
%Find the area under the curve.
% \[
% \int_{0.5}^1 x(1-x)^2 dx
% \] \pause
%$$g(x) = x, \hspace{.3in} h^{'}(x) = (1-x)^{2}$$ \pause
%$$g^{'}(x) = 1, \hspace{.3in} \pause h(x) =\int (1-x)^{2} dx$$ \pause
%
%Let $u = 1-x$.  Then $du = -1 \cdot dx$.  We can write the integral
%$$h(x) = \int (1-x)^{2} dx = \int u^{2} dx = \int u^{2} (-1) du = -1 \cdot \frac{u^{3}}{3} = \frac{-1}{3} \cdot (1-x)^{3}$$
%
%\end{frame}
%
%
%
%\begin{frame}{Integration by Parts}{Beta Distribution Example}
%
%Now we have all the pieces we need to apply the integration by parts formula.
%$$g(x) = x, \hspace{.3in} h^{'}(x) = (1-x)^{2}$$
%$$g^{'}(x) = 1, \hspace{.3in} h(x) = \frac{-1}{3}  (1-x)^{3}$$ 
%\pause
%\begin{eqnarray*}
% \int_{0.5}^1 x(1-x)^2 dx &=  \left. \left[ \frac{-1}{3} (1-x)^{3} \cdot x - \int \frac{-1}{3} (1-x)^{3} \cdot 1 dx \right] \right|_{0.5}^{1} \\
% &=  \left. \left[ \frac{-1}{3} x(1-x)^{3}  + \frac{1}{3} \int (1-x)^{3}  dx \right] \right|_{0.5}^{1} \\
% \end{eqnarray*}
%\end{frame}
%
%
%\begin{frame}{Integration by Parts}{Beta Distribution Example}
%\begin{eqnarray*}
% \int_{0.5}^1 x(1-x)^2 dx &=  \left. \left[ \frac{-1}{3} (1-x)^{3} \cdot x - \int \frac{-1}{3} (1-x)^{3} \cdot 1 dx \right] \right|_{0.5}^{1} \\
% &=  \left. \left[ \frac{-1}{3} x(1-x)^{3}  + \frac{1}{3} \int (1-x)^{3}  dx \right] \right|_{0.5}^{1} \\
% \end{eqnarray*}\pause
%To do the integral in the last term:
%$ \int (1-x)^{3} dx$
%Let $u = 1-x$.  Then $du = -1 \cdot dx$.  We can write the integral
%$$\int (1-x)^{3} dx = \int u^{3} dx = \int u^{3} (-1) du = -1 \cdot \frac{u^{4}}{4} = \frac{-1}{4} \cdot (1-x)^{4}$$
%\end{frame}
%
%
%\begin{frame}{Integration by Parts}{Beta Distribution Example}
%Putting this back into our equation
%\begin{eqnarray*}
% \int_{0.5}^1 x(1-x)^2 dx  &=  \left. \left[ \frac{-1}{3} x(1-x)^{3}  + \frac{1}{3} \int (1-x)^{3}  dx \right] \right|_{0.5}^{1} \\ \pause
% &=  \left. \left[ \frac{-1}{3} x(1-x)^{3}  + \frac{1}{3} \left(  \frac{-1}{4}  (1-x)^{4} \right) \right] \right|_{0.5}^{1} \\ \pause
% &=  \left. \left[ \frac{-1}{3} x(1-x)^{3}  - \frac{1}{12} (1-x)^{4} \right] \right|_{0.5}^{1} \\ \pause
% &=  [ 0 - 0 ]  - \left[ \frac{-1}{3} (0.5)(1-0.5)^{3}  - \frac{1}{12} (1-0.5)^{4} \right]  \\ \pause
%&=  \frac{1}{3} 0.5^{4}  + \frac{1}{12} 0.5^{4} = \frac{5}{192}
% \end{eqnarray*}
%\end{frame}
%


\begin{frame}{The End}
Questions?

\end{frame}



\end{document} % don't forget this
